
Currently the algorithm handles uncertainties in position, but not in the
environment, and must therefore be run in a known environment, and is hence a
strictly off-line motion planner. However, it is possible to generalize the
algorithm to handle unknown environments.

In general, the funnels generated are tight outer approximations of the true
reachable set for the system. However, note that since the Lyapunov function
employed is quadratic, it will always be symmetric around the trajector
verified. This means that even though the real nonlinear system dynamics can
have a tight reachable set on one side of the trajector, the symmetry of the
quadratic Lyapunov function might lead the funnel to be too conservative on one
side of the trajector. For this paper however, the system dynamics are
symmetrical, so this was not an issue. That the funnels generated do provide
tight outer approximations was shown verified through a Monte-Carlo simulation
of the nonlinear system with bounded uncertainty. Over the course of \(100\)
simulations the system never left the funnels, and the funnels were shown to
provide a tight outer approximation.


The strength of the algorithm lies in that it can separate handling the
uncertainty into an off-line pre-computation phase. Therefore, the global motion
planner does not need to be significantly more complex than if it had not
taken uncertainty into account. In fact, it can remain oblivious to the
overarching problem difficulty of handling uncertainties for a complex nonlinear
system. In fact, once the motion primitives have been calculated and verified
off-line, they might as well be employed in any global motion planner able to
handle discrete motion primitives.


Even though the robustness guarantees of the SOS framework could not be
guaranteed for an entire off-line phase, due to the planner not handling
multiple controller inputs, and hence the funnels did not compose off-line.
Still, the \rrtfunnel{} algorithm did run-time verification of funnel abidance.
Even though the model leaving a non-composable funnel at run-time was
theoretically possible, this did not cause the airplane to collide a single time
over the course of \(300\) simulation runs. This was in starch contrast to the
benchmark planner, which did not handle uncertainty at all, and instead relied
on avoiding the obstacles by as big a margin as possible, and therefore
consistently had a collision rate of \(6\%\) or higher. This collission rate can
be expected to be a lot higher in a denser planning environment, but this would
also significantly add to the time of the off-line planning phase. As can be
seen in the last column, when the cross-wind added to the experiment violated
the upper bound of \(3\) \IEEEunit{m/s}, the \rrtfunnel{} algorithm did also
start crashing. It is seen that the
\rrtfunnel{} algorithm performs better in terms of robustness up to and
including its uncertainty bound of \( w = 3 \) \IEEEunit{m/s}, while the
benchmark planner does collide in the same environment, with the same
uncertainty.

Although the \rrtfunnel{} algorithm performed better in terms of collisions, it
does a lot worse in terms of exploring the planning space, than does the
benchmark planner. This only increases with the difficulty of the planning
space, and hence the time spent by the \rrtfunnel{} algorithm is significantly
longer.
