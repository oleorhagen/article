
This paper has developed and shown that the \rrtfunnel{} motion planning
algorithm does provide robust feedback motion planning for nonlinear dynamical
systems. Although, the current implementation cannot give lent to the proof that
the funnel compositions in general are provably safe, due to the \ac{SOS}
formulations employed do not handle input saturations and multiple controller
inputs. However, this can be added given a few extra conditions on the \ac{SOS}
optimization problem, and will be referred to as further work.


Currently the algorithm handles uncertainties in position, but not in the
environment, and must therefore be run in a known environment, and is hence a
strictly off-line motion planner. This is a known limitation of the \ac{RRT}
algorithm, and methods have been devised for amending this limitation. However,
none of them have been implemented with the \rrtfunnel{} algorithm at present.


In general, the funnels generated are tight outer approximations of the true
reachable set for the system. However, note that since the Lyapunov function
employed is quadratic, it will always be symmetric around the trajector
verified. This means that even though the real nonlinear system dynamics can
have a tight reachable set on one side of the trajector, the symmetry of the
quadratic Lyapunov function might lead the funnel to be too conservative on one
side of the trajector. For this paper however, the system dynamics are
symmetrical, so this was not an issue. That the funnels generated do provide
tight outer approximations was shown to be true through a Monte-Carlo simulation
of the nonlinear system with bounded uncertainty. Over the course of \(10.000\)
simulations the system never left the funnels, and the funnels were shown to
provide a tight outer approximation.


The strength of the algorithm lies in that it can separate handling the
uncertainty into an off-line pre-computation phase. Therefore, the global motion
planner does not need to be significantly more complex than if it had not
reasoned about uncertainty at all. In fact, it can remain oblivious to the
overarching problem difficulty of handling uncertainties for a complex nonlinear
system. In fact once the motion primitives have been calculated and verified
off-line, they might as well be employed in any global motion planner able to
handle discrete motion primitives.


Even though the robustness guarantees of the \ac{SOS} framework were lost in the
experiments section, due to the planner not handling multiple controller inputs,
the \rrtfunnel{} algorithm did not cause the airplane to collide a single time
over the course of \(300\) simulation runs. This was in starch contrast to the
benchmark planner, which did not handle uncertainty at all, and instead relied
on avoiding the obstacles by as big a margin as possible, and therefore
consistently had a collision rate of \(25\%\) or higher. Therefore, not only was
the \rrtfunnel{} planner able to provide safer traversal of the experiment
environment, but it could also provide more direct traversal, as the distance to
the trees did not matter, and more aggressive maneuvers could be executed
freely.

% The implementation of the algorithm in this paper leveraged the discrete
% sampling of points in order to generate funnels around these discrete points.
% However, a continuous approach might have been better suited, as the time for
% generating the funnels off-line are not important for the algorithm at runtime.
