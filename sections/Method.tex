This section will introduce and develop the \rrtfunnel{} algorithm, by two
means: First develop robust motion primitives through the \ac{SOS} programming
framework based on the work by \cite{majumdarFunnelLibrariesRealtime2017},
and second, deploy these funnels as robust motion primitives in a discrete
\ac{RRT} robust motion planner based on \cite{Lav06}. Using robust motion
primitives has several advantages. Firstly, they handle uncertainty, and thus,
as long as the uncertainties in the system are akin to the assumptions on the
incoming uncertainty parameters, the dynamical system will not leave the funnel,
and hence if the funnel is not in collision, neither will the system be.
Secondly, as the motion primitives are robust, there is no need for more
conservative maneuvers and heuristics, such as maximizing the distance to an
obstacle, which many motion planners do in order to handle uncertainty. Since
the primitives are robust, the system might as well choose a primitive that is
close to an obstacle, as one that is far away, since the funnel is guaranteed to
be collision-free in both cases. This means that a robust motion algorithm can
perform more aggressive maneuvers than one that is inherently conservative about
its environment and maneuvers~\cite{singhRobustOnlineMotion2017}.


\subsection{Generating Robust Motion Primitives}
\label{sec:generating-robust-motion-primitives}

In order to create a basic set of funnels as the robust motion primitives for
the \rrtfunnel{} motion planning algorithm, a few obstacles has to be overcome.
The first one is settling on a dynamical system model for the funnel
calculations. This thesis employs the simple unicycle model from
\author{Lav06}~\cite[613]{Lav06} which is modified slightly into
\begin{equation}
  \label{eq:model-dynamics}
  \vect{x} =
  \begin{bmatrix}
    x \\ y \\ \theta \\
  \end{bmatrix}, \, \dot{\vect{x}} =
  \begin{bmatrix}
    -v \sin(\theta) \\
    v \cos(\theta) \\
    u \\
  \end{bmatrix}
  ,
\end{equation}
which is a first-order unicycle model with constant speed of
\(v=10\) \IEEEunit{m/s}. A picture of the model can be found in
\cref{fig:second-order-unicycle}.
\begin{figure}[!t]
  \def\svgwidth{\columnwidth}
  \import{figures/method/}{unicycle-model.pdf_tex}
  \caption{The unicycle model of an airplane.}
  \label{fig:second-order-unicycle}
\end{figure}
Although this is the only model used in this thesis, the framework and the code
is easily adapted into accommodating a different and more complicated model.