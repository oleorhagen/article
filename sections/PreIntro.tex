
\section{Introduction}
\IEEEPARstart{M}{otion} planning concerns the problem of finding a dynamically
feasible path from an initial configuration to a defined end state in a safe
manner. In order for a motion planner to handle real world motion planning tasks
it needs to handle the uncertainty that comes with a real-life planning problem.
This is especially difficult for nonlinear dynamical systems. Knowledge of
position, the environment and the dynamics of the system are all uncertain to
some degree. Sensory noise, tuning and readings may be off. Limited precision,
and accidents may hinder the measurements and leave them with an error term.
Sensory noise, tuning and readings may be off. Thus in order for a planner to
guarantee safe traversal through a real world environment, a motion planner
needs to handle uncertainties.

In the face of uncertainty some planners choose to ignore these error sources
and instead apply heuristics such as maximizing the distance to the obstacles in
the environment. However, this adds the disadvantage that the plans can become
overly conservative. Explicitly handling the uncertainties in the planning stage
enables the planner to employ more aggressive maneuvers, such as going through
two obstacles that are close together, as opposed to going around the difficult
area. If uncertainties are accounted for, going straight through is an
acceptable maneuver for a planner that has guarantees on the whereabouts of the
dynamical system, and hence is not afraid to get close to an obstacle.

\subsection{Literature Review}


\subsubsection{Robust Motion Planning}

This work is a part of the broader field of planning under uncertainty. The
algorithms are further classified depending on classes of uncertainty they can
handle. That is, uncertainty in the planning environment, uncertainty in the
model, uncertainty in sensing, and uncertainty in the map.

\subsubsection{RRT with Robust Motion Primitives}

On the topic of combining the \textit{RRT} algorithm with robust motion
primitives, Tedrake~\cite{tedrakeLQRtreesFeedbackMotion2009} creates a tree of
robust LQR controllers, in order to guarantee safe traversal of the dynamical
system at hand. Luders generate robust motion primitives in real-time for an
\textit{RRT*} algorithm through chance constraints in~\cite{luders2013robust}.
Another robust extension to the \textit{RRT} algorithm can be found in
Melchior~\cite{melchior2007particle}, where each extension to the tree is
treated as a stochastic process, and simulated multiple times, and since pruned
based on the expected probability of successful execution.


\subsubsection{SOS and Robust Motion Planning}

The research done on the topic of merging \textit{SOS} theory with motion
planning is not yet vast, but interesting due to its general robustness
guarantees, with no probability involved. In
Tedrake~\cite{tedrakeLQRtreesFeedbackMotion2009} a tree of LQR-controllers is
created, in order to safely take a dynamical system from one intial state to a
final state. In~\cite{majumdarFunnelLibrariesRealtime2017}, Majumdar seeks to
limit the size of the area in which a controller will take a dynamic system.


This paper builds upon the work done on verifying regions of attraction for
nonlinear dynamical systems through the use of Lyapunov functions which are
verified for a polynomial system through the use of SOS programming.

\subsection{Contributions}

This paper combines the theory of verified reachable sets, referred to as
funnels, with a discrete RRT motion planner in order to provide provably robust
motion planning for a nonlinear dynamical system. To the best of the authors'
knowledge, this is the only known RRT algorithm which implements verified
reachable sets as discrete motion primitives as extension operators for the a
RRT algorithm.

The result of this work is the RRT-Funnel motion planning algorithm, a discrete
motion planning algorithm which can guarantee safe passage through an obstacle
space, given the assumption that the uncertainty in the system is bounded. The
funnel definitions in this paper is taken from a series of articles on
funnels~\cite{Tobenkin_2011,tedrakeLQRtreesFeedbackMotion2009,
  majumdarRobustOnlineMotion2013,
  majumdarFunnelLibrariesRealtime2017,ahmadi2014dsos}, with the main focus being
on \cite{majumdarFunnelLibrariesRealtime2017}.



