
\section{Introduction}
\IEEEPARstart{I}{n} order for a motion planner to handle real world motion
planning tasks it needs to handle the uncertainty that comes with a real-life
planning problem. This is especially difficult for nonlinear dynamical systems.
Knowledge of position, the environment and the dynamics of the system are all
uncertain to some degree. Sensory noise, tuning and readings may be off. Limited
precision, and accidents may hinder the measurements and leave them with an
error term. Thus in order for a planner to guarantee 2 Actual Planned safe
traversal through a real world environment, a motion planner needs to handle
uncertainties.Knowledge of position, the environment and the dynamics of the
system are all uncertain to some degree. Sensory noise, tuning and readings may
be off. Limited precision, and accidents may hinder the measurements and leave
them with an error term. Thus in order for a planner to guarantee 2 Actual
Planned safe traversal through a real world environment, a motion planner needs
to handle uncertainties.

In this case a lot of planners simply choose to ignore these error sources and
apply heuristics such as maximizing the distance to the obstacles in the
environment. However, this adds the disadvantage that the plans can become
overly conservative. Explicitly handling the uncertainties in the planning stage
enables the planner to employ more aggressive maneuvers, such as flying through
two trees close together, as opposed to flying around the entire grove. Flying
straight through is an acceptable maneuver for a planner that has guarantees on
the whereabouts of the dynamical system, and hence is not afraid to go close to
an obstacle. This confidence in planning is enabled by the calculation of robust
motion primitives through the SOS framework developed in \cref{sec:Method}.

As it currently stands the current implementation of the algorithm handles
uncertainty only in the position of the dynamical system, but can be easily ex-
tended to handle uncertainty in input and speed as well by employing a different
dynamical model, as well as enabling the handling of input saturations, which is
currently not supported.


\subsection{Contributions}

This paper merges the theory of verified reachable sets, referred to as funnels,
with a discrete \ac{RRT} motion planner in order to provide provably robust
motion planning for a nonlinear dynamical system. To the best of the author's
knowledge, this is the only known \ac{RRT} algorithm which implements verified
reachable sets as discrete motion primitives as extension operators for the
algorithm.

The result of this work is the RRT-Funnel motion planning algorithm, a discrete
motion planning algorithm which can guarantee safe passage through an obstacle
space, given assumptions on the uncertainty in the system.

Currently the algorithm has problems with the funnel composition, but this is
amenable through adding input saturations to the funnel calculations, and is
referred to as future work. Thus in the case of the vehicle during simulation
not being able to enter the next funnel in a chain, the algorithm will execute
an 'emergency maneuver', which in this case is a stop.

\subsection{Building Blocks}

This paper builds upon the work done on verifying regions of attraction for
nonlinear dynamical systems through the use of Lyapunov functions which are
verified for polynomial systems through the use of \ac{SOS} programming. These
regions of attraction are referred to as funnels, and first appeared in the
literature in \cite{masonMechanicsManipulation1985}, but is later employed in a
lot of research. The funnel definitions in this thesis is taken from a series of
articles on funnels~\cite{Tobenkin_2011,tedrakeLQRtreesFeedbackMotion2009,
  majumdarRobustOnlineMotion2013,
  majumdarFunnelLibrariesRealtime2017,ahmadi2014dsos}, with the main focus being
on \cite{majumdarFunnelLibrariesRealtime2017}.

The optimization problems obtained are in general non-convex, as the
formulations are bilinear in the decision variables, and are solved through
bilinear alternation. As such there is no guarantee of a globally optimal
solution, and hence feasible initializations are required in order for the
program to converge.

\subsection{Problem Statement}

Given the nonlinear dynamical system
\begin{equation}
  \dot{\vect{x}} = f\big(\vect{x}(t), \vect{u}(t), w(t) \big), \label{eq:dynamicalsystem}
\end{equation}
with state \(\vect{x}(t) \in \R^3\), \(\vect{u}(t) \in \R\), and \(w(t) \in \R\). Then create
a set of robustly verified discrete motion primitives through the use of
\ac{SOS} programming to verify the reachable set for each of the trajectories in
a base set of trajectories \(T_{0}\). Next, apply these trajectories, with the
robust reachable set surrounding them as motion primitives for a discrete
\ac{RRT} motion planning algorithm and compare the results to a \ac{RRT}
algorithm which does not employ the reachable sets around the base trajectories,
but instead relies on maximizing the distance to the nearest obstacles, in order
to handle uncertainty.

