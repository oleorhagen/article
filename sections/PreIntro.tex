
\section{Introduction}
\IEEEPARstart{M}{otion} planning concerns the problem of finding a dynamically
feasible path from an initial configuration to a defined end state in a safe
manner. In order for a motion planner to handle real world motion planning tasks
it needs to handle the uncertainty that comes with a real-life planning problem.
This is especially difficult for nonlinear dynamical systems. Knowledge of
position, the environment and the dynamics of the system are all uncertain to
some degree. Sensory noise, tuning and readings may be off. Limited precision,
and accidents may hinder the measurements and leave them with an error term.
Sensory noise, tuning and readings may be off. Thus in order for a planner to
guarantee safe traversal through a real world environment, a motion planner
needs to handle uncertainties.

In the face of uncertainty some planners choose to ignore these error sources
and instead apply heuristics such as maximizing the distance to the obstacles in
the environment. However, this adds the disadvantage that the plans can become
overly conservative. Explicitly handling the uncertainties in the planning stage
enables the planner to employ more aggressive maneuvers, such as going through
two obstacles that are close together, as opposed to going around the difficult
area . If uncertainties are accounted for, going straight through is an
acceptable maneuver for a planner that has guarantees on the whereabouts of the
dynamical system, and hence is not afraid to get close to an obstacle.

\subsection{Contributions}

This paper combines the theory of verified reachable sets, referred to as
funnels, with a discrete RRT motion planner in order to provide provably robust
motion planning for a nonlinear dynamical system. To the best of the author's
knowledge, this is the only known RRT algorithm which implements verified
reachable sets as discrete motion primitives as extension operators for the
algorithm.

The result of this work is the RRT-Funnel motion planning algorithm, a discrete
motion planning algorithm which can guarantee safe passage through an obstacle
space, given the assumption that the uncertainty in the system is bounded.

This paper builds upon the work done on verifying regions of attraction for
nonlinear dynamical systems through the use of Lyapunov functions which are
verified for polynomial systems through the use of SOS programming. These
regions of attraction are referred to as funnels, and first appeared in the
literature in \cite{masonMechanicsManipulation1985}, but is later employed in a
lot of research. The funnel definitions in this paper is taken from a series of
articles on funnels~\cite{Tobenkin_2011,tedrakeLQRtreesFeedbackMotion2009,
  majumdarRobustOnlineMotion2013,
  majumdarFunnelLibrariesRealtime2017,ahmadi2014dsos}, with the main focus being
on \cite{majumdarFunnelLibrariesRealtime2017}.

\subsection{Problem Statement}

Given the nonlinear dynamical system
\begin{equation}
  \dot{\vect{x}} = f\big(\vect{x}(t), \vect{u}(t), w(t)
  \big), \label{eq:dynamicalsystem}
\end{equation}
with state \(\vect{x}(t) \in \R^3\), \(\vect{u}(t) \in \R\), and \(w(t) \in
\R\). Then create a set of robustly verified discrete motion primitives through
the use of SOS programming to verify the reachable set for each of the
trajectories in a base set of trajectories \(T_{0}\). Next, apply these
trajectories, with the robust reachable set surrounding them as motion
primitives for a discrete RRT motion planning algorithm and compare the results
to a RRT algorithm which does not employ the reachable sets around the base
trajectories, but instead relies on maximizing the distance to the nearest
obstacles, in order to handle uncertainty.

\subsection{Survey of Papers}

This paper is similar in spirit to~\cite{tedrakeLQRtreesFeedbackMotion2009}
where a tree of LQR-controllers is created, in order to safely take a dynamical
system from one intial state to a final state. The difference however is that in
this paper the funnels are computed off-line, and are employed as motion
primitives to a RRT-planner, in addition the LQR-trees algoritm plans backwards,
while this implementation plans forwards in time. The funnel generation is also
slightly different, as the funnels in this paper, are based off of the
formulation in~\cite{majumdarFunnelLibrariesRealtime2017}, which also seeks to
limit the size of the funnels. Other approaches that are similar in spirit to
this implementation is~\cite{lenySequentialCompositionRobust2012}, as the paper
also employs an RRT algorithm which builds a tree of funnels. The differences
however are that this algorithm is discrete and the funnels are computed using a
SOS framework.

