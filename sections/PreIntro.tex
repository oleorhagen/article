
\section{Introduction}

%\IEEEPARstart{M}{otion} planning concerns the problem of finding a
Motion planning concerns the problem of finding a
dynamically feasible path from an initial configuration
to a defined end state in a safe manner. In order for a motion
planner to handle real world motion planning tasks it needs
to handle the uncertainty that comes with a real-life planning
problem. This is especially difficult for nonlinear dynamical
systems. Knowledge of the system’s state, the environment and
the dynamics of the system are all uncertain to some degree.
Limited measurement precision and use of imperfect world
and vehicle models will therefore leave error terms in the
state and world estimations. Thus, in order for a planner to
guarantee safe traversal through a real world environment, a
motion planner needs to handle uncertainties.


In the face of uncertainty, some planners choose to ignore these error sources
and instead apply heuristics such as maximizing the distance to the obstacles in
the environment or setting a fixed safety radius~\cite{hoyAlgorithmsCollisionfreeNavigation2015}. However, this adds the
disadvantage that the plans can become overly conservative. Explicitly handling
the uncertainties in the planning stage enables the planner to employ more
aggressive maneuvers, such as going through two obstacles that are close
together, as opposed to going around the difficult area. If uncertainties are
accounted for, going straight through is an acceptable maneuver for a planner
that has guarantees on the whereabouts of the dynamical system, and hence is not
afraid to get close to an obstacle. This means that a robust motion algorithm
can perform more aggressive maneuvers than one that is inherently conservative
about its environment and maneuvers~\cite{singhRobustOnlineMotion2017}.

In motion planning, there is a separation between global
and local methods~\cite{hoyAlgorithmsCollisionfreeNavigation2015}, where global methods assumes a known
map of the environment and local methods assumes only local
knowledge. Thus, global methods can plan paths from the start
state to the goal, whereas local methods can only plan a part
of the way. As local methods explicitly consider the end goal
during planning, they are more susceptible to getting trapped
in local minima.

Previous efforts to handle uncertainties in global path
planning include using the Rapidly-exploring Random Tree
(RRT) algorithm with robust motion primitives. This approach
is taken by Tedrake~\cite{tedrakeLQRtreesFeedbackMotion2009}, where a tree consisting of robust
Linear-Quadratic Regulator (LQR) controllers are created, in
order to guarantee safe traversal of the planning environment.
Van den Berg [4] has a similar approach of creating motion
primitives where a LQR is used, and a Gaussian process
is used to model uncertainty, thus allowing robust control
along with an uncertain system. Luders generate robust motion
primitives in real-time for an RRT* algorithm through chance
constraints in~\cite{luders2013robust}, but also ensures asymptotic optimality.
Another robust extension to the RRT algorithm can be found
in Melchior~\cite{melchior2007particle}, where each extension to the tree is treated
as a stochastic process, and simulated multiple times, and
since pruned based on the expected probability of successful
execution.

A recent approach to the local path planning problem
has been to use Sum of Squares (SOS) theory with motion
planning. As mentioned above, this is the approach taken by
Tedrake in~\cite{tedrakeLQRtreesFeedbackMotion2009}. In~\cite{majumdarFunnelLibrariesRealtime2017}, Majumdar expands upon this idea and
seeks to limit the size of the area in which a controller will take
a dynamic system, while at the same time giving guarantees
of safe traversal. This enables real-time motion planning in
highly complex environments with advanced vehicle models,
as long as the controllers are generated off-line. The offline generation is a requirement due to the time they take
to generate. The current research on the topic is limited, but
interesting due to the formal general robustness guarantees it
provides.

However, the method is a local method, and thus has a
limited planning horizon. In~\cite{majumdarFunnelLibrariesRealtime2017}, a simulated experiment with
large-scale environments is conducted, but only by repeating
the local process several times. If we instead can use the theory
and framework introduced here and combine it with a global
planner, we can achieve longer planning horizons and thereby
reduce likelihood of getting trapped in local minima .

This paper builds upon the work done on verifying regions of attraction —
referred to as funnels — for nonlinear dynamical systems through the use of
Lyapunov functions. Verification is achieved for a polynomial system through the
use of SOS programming. We combine the theory of funnels with a discrete RRT
motion planner in order to provide a provably robust global motion planner for a
nonlinear dynamical system. To the best of the authors’ knowledge, this is the
only known RRT algorithm which implements verified reachable sets generated
through SOS programming as discrete motion primitives as extension operators for
the an RRT algorithm. The novel method is then compared with a RRT algorithm
that maximize the distance to the nearest obstacles, in order to show which
handles the planning task best. The result of this work is the RRT-Funnel motion
planning algorithm, a discrete motion planning algorithm which can guarantee
safe passage through an obstacle space, given the assumption that the
uncertainty in the system is bounded. The funnel definitions in this paper is
taken from a series of articles on
funnels~\cite{Tobenkin_2011,tedrakeLQRtreesFeedbackMotion2009,
  majumdarRobustOnlineMotion2013,
  majumdarFunnelLibrariesRealtime2017,ahmadi2014dsos}, with the main focus being
on~\cite{majumdarFunnelLibrariesRealtime2017}.
