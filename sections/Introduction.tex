\section{Method}

\subsection{Funnels}
\label{sec:funnels}

The uncertainty guarantees in this paper is given through creating
\textit{funnels}. Funnels are the parameterization of the \textit{finite time
  reachable sets} for the dynamical system at hand. The following sections will
introduce and develop the theory needed to understand the \ac{SOS} framework
that lies at the bottom of the mathematical verification of these reachable
sets. 

A \textit{funnel} is a parameterization of the reachable set of a dynamical
system. This means that a Funnel holds all the states the dynamical system can
be in during a planning task. Mathematically the reachable set of the system is
defined as
\[
  \vect{x}(0) \in \mathcal{X}_0 \implies \vect{x}(t) \in F(t), \forall t \in
  \sqb{0,T},
\]
where \(\mathcal{X}_0\) is the set of initial conditions, \(\sqb{0,T}\) the time
interval, and \(F(t)\) is the set of states that the system can be in at time
\(t\) \cite{majumdarFunnelLibrariesRealtime2017}. Although this paper concerns
itself with approximating the reachable set through \textit{Lyapunov} functions,
a useful analogy is imagining the funnel created through a \textit{Monte-Carlo}
simulation, where the funnel is the set of all the paths traversed by the
dynamical system at hand.

In the literature, the term funnel first appears in
\cite{masonMechanicsManipulation1985}, but is later employed in a lot of
research. The funnel definitions in this paper is taken from a series of
articles on funnels~\cite{Tobenkin_2011,tedrakeLQRtreesFeedbackMotion2009,
  majumdarRobustOnlineMotion2013,
  majumdarFunnelLibrariesRealtime2017,ahmadi2014dsos}, with the main focus being
on \cite{majumdarFunnelLibrariesRealtime2017}.


\subsection{Funnel Composition}

Now that discrete motion primitives in the form of funnels can be calculated for
a given base set of trajectories, it is time to handle the overarching goal of
creating a plan consisting of multiple funnels stacked from start to finish, so
that safe traversal can be guaranteed along the given path. In order for two
funnels to create one extended motion primitive from multiple smaller
primitives, the funnels in use must be composable. In order for two funnels to
be composable, the outlet of one funnel needs to be completely contained within
the inlet of the other. This means that if \(\mathcal{F}_1 = F_1(T)\) is the
outlet of funnel \(F_1\), and \(\mathcal{F}_2 = F_2(0)\) is the inlet of
\(F_2\), then
\begin{equation}
  \label{eq:funnel-subset}
  \mathcal{F}_1 \subseteq \mathcal{F}_2
\end{equation}
is required in order for the funnels to be composable.
\begin{definition}
  \label{def:funnel-composition}
  An ordered pair \((F_{1}, F_{2})\) of funnels \(F_1 \colon [0,T_1] \rightarrow
  \mathcal{P}(\R^n)\) and \(F_2 \colon [0,T_2] \rightarrow \mathcal{P}(\R^n)\)
  are sequentially composable if \(F_1(T) \subseteq F_2(0)\).
\end{definition}
Because the implication
\begin{equation}
  V_1(T_1,\bar{\vect{x}}) \leq \rho_1(T_1) \implies V_2(0,\bar{\vect{x}}) \leq
  \rho_2(0)
\end{equation}
is an equivalent condition to \cref{eq:funnel-subset}, and because it can be
checked through the following \ac{SOS} program
\begin{IEEEeqnarray*}{lC}
  \text{Find } \; L(\bar{x}) \IEEEyesnumber \\
  \text{s.t.} \\
  \IEEEeqnarraymulticol{2}{l}{\rho_2(0) - V_2( 0,\bar{\vect{x}}) - L(\bar{\vect{x}})
                  \bigl( \rho_1(T_1) - V_1(T_1,\bar{\vect{x}}) \bigr) \quad \text{is SOS}} \mathEoS \nonumber
\end{IEEEeqnarray*}
Hence funnel composition can be checked in the same way that the funnels are
generated -- through a \ac{SOS} program. Also note that the above program is a
simple search for existence, and hence no cost function is needed.

However in
\citeauthor{majumdarFunnelLibrariesRealtime2017}~\cite{majumdarFunnelLibrariesRealtime2017},
the program is stated simply, and it can be helpful to take a look at the
derivation in order to gain a feel for the implementation of a \ac{SOS} program.

The \nameref{sec:s-procedure} enables the search to be limited to a
semi-algebraic set. In this case, that set is \(\mathcal{F}_2 = \set{\vect{x}
  \in \R^n \mid V_2(0,\bar{\vect{x}}) \leq \rho_2(0)}\), and any \(\vect{x}\)
that is not in this set is therefore not of importance. In more general terms
this can be written
\begin{equation}
  \label{eq:example-s-procedure-implication}
  \vect{x} \in \mathcal{F}_2 \implies p(\vect{x}) \geq 0,
\end{equation}
where \(p(\vect{x})\) is the \ac{SOS} polynomial that is to be verified. In this
case \(p(\vect{x})\) is \(V_2(0,\bar{\vect{x}})\). Thus in order to impose the
implication~\eqref{eq:example-s-procedure-implication} define
\begin{equation}
  q(\vect{x}) = V_2(0,\bar{\vect{x}}) - \rho_2(0) - L(\bar{\vect{x}}) \bigl(
    \rho_1(T_1) - V_1(T_1,\bar{\vect{x}}) \bigr),
\end{equation}
where \(q(\vect{x})\) and \(L(\bar{\vect{x}})\) need to be SOS polynomials.
